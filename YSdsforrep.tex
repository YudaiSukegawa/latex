\NeedsTeXFormat{LaTeX2e}
\ProvidesPackage{YSdsforrep}
%
\RequirePackage{luatexja}
\RequirePackage[explicit,pagestyles]{titlesec}
\RequirePackage{titletoc}
\RequirePackage[truedimen,margin=0.14\paperwidth]{geometry}
\RequirePackage[luatex]{graphicx}
\RequirePackage[luatex,table,hyperref]{xcolor}
\RequirePackage{amssymb,mathtools}
\RequirePackage{float}
\RequirePackage[unicode=true,linktoc=all,%
bookmarksnumbered=true,bookmarksopen=true,%
colorlinks=true,linkcolor=blue,citecolor=red]{hyperref}
\RequirePackage[nameinlink]{cleveref}
\RequirePackage{caption}

\RequirePackage{tcolorbox,tikzlings}
\tcbuselibrary{raster,skins,xparse,breakable}
\usetikzlibrary{shadings,calc}
\RequirePackage{varwidth}
%
\DeclareTColorBox{shapebox2}{ O{blue} m d"" !O{} }{enhanced,colframe=#1!50!black,colback=#1!10!white,
	colbacktitle=#1!5!yellow!10!white, fonttitle=\large\bfseries,coltitle=black,IfValueTF={#3}{watermark text={#3}}{},
	attach boxed title to top center= {yshift=-0.25mm-\tcboxedtitleheight/2,yshifttext=2mm-\tcboxedtitleheight/2}, 
	boxed title style={boxrule=0.5mm, frame code={ \path[tcb fill frame] ([xshift=-6mm]frame.west) -- (frame.north west) -- (frame.north east) -- ([xshift=6mm]frame.east) -- (frame.south east) -- (frame.south west) -- cycle; }, 
		interior code={ \path[tcb fill interior] ([xshift=-3mm]interior.west) -- (interior.north west) -- (interior.north east) -- ([xshift=3mm]interior.east) -- (interior.south east) -- (interior.south west) -- cycle;} },
	title={#2},#4}
\newcommand{\PartialToc}{\vspace{1em}\begin{center}
		\normalsize\normalfont{\Large\bfseries 目次}\smallskip
		\startcontents[chapters]
		\printcontents[chapters]{}{1}[2]{}
\end{center}\vspace{-3em}}
\newif\if@link
\@linkfalse
\newif\if@ptc
\@ptcfalse
%
\DeclareOption{withlink}{\@linktrue}
%
\DeclareOption{sec=normal}{
	\titleformat{\chapter}[display]{\huge\bfseries}{第\thechapter 章}{1ex}{#1}[]
	\titleformat{\section}[hang]{\normalfont\large\bfseries}{\thesection}{1em}{#1\hfill\if@link \hyperlink{mokuji}{\color{magenta}目次へ戻る}\else \fi}[]
	\titleformat{name=\section,numberless}[hang]{\normalfont\large\bfseries}{#1}{0pt}{\hfill\if@link \hyperlink{mokuji}{\color{magenta}目次へ戻る}\else \fi}[]
}
%
\DeclareOption{sec=1}{

\titlespacing{\section}{0pt}{-2em}{-2em}
\titlespacing{\subsection}{0pt}{-2em}{-2em}
}
%
\DeclareOption{sec=2}{
\titleformat{\chapter}[hang]{\huge\bfseries}{\begin{tikzpicture}
		\pgfmathparse{100*rnd}
		\edef\tmp{\pgfmathresult}
		\pgfmathparse{100*sin(rnd*pi/2 r)}
		\edef\tmq{\pgfmathresult}
		\draw[line width=5pt,color=cyan!\tmp!magenta!\tmq!yellow]
		(0,0) -- (\linewidth,0) (0,3.8) -- (\linewidth,3.8);
		\draw[line width=2pt,color=cyan!\tmp!magenta!\tmq!yellow]
		(0,2) -- (\linewidth,2);
		\node at (\linewidth/2,1) {#1};
		\node at (\linewidth/2,2.8) {第\thechapter 章};
		\node at (\linewidth-2em,0.5) {\footnotesize\if@link \hyperlink{mokuji}{\color{magenta}目次へ戻る}\else \fi};
\end{tikzpicture}}{-1em}{\hypertarget{mokuji\thechapter}{}}[
\if@ptc{\PartialToc}\else \fi]
\titleformat{\section}[hang]{\Large\bfseries}{\begin{tikzpicture}
		\pgfmathparse{100*rnd}
		\edef\tmp{\pgfmathresult}
		\pgfmathparse{100*sin(rnd*pi/2 r)}
		\edef\tmq{\pgfmathresult}
		\filldraw[line width=3pt,draw=cyan!\tmp!magenta!\tmq!yellow,fill=white,rounded corners] (0,0) rectangle (\linewidth-3pt,1);
		\tikzling[xshift=1.5em,yshift=0.1mm,scale=0.4,graduate=black,tassel=red]
		\node at (\linewidth/2,0.5) {\thesection~~#1};
		\node at (\linewidth-2em,0.5) {\footnotesize\if@link \hyperlink{mokuji\thechapter}{\color{magenta}章の最初に戻る}\else \fi};
	\end{tikzpicture}
}{-7pt}{}[]
\titleformat{\subsection}[hang]{\normalfont\large\bfseries}{}{0pt}{
\thesubsection\quad#1
}[]
\titleformat{name=\section,numberless}[hang]{\Large\bfseries}{\begin{tikzpicture}
		\pgfmathparse{100*rnd}
		\edef\tmp{\pgfmathresult}
		\pgfmathparse{100*sin(rnd*pi/2 r)}
		\edef\tmq{\pgfmathresult}
		\filldraw[line width=3pt,draw=cyan!\tmp!magenta!\tmq!yellow,fill=white,rounded corners] (0,0) rectangle (\linewidth-3pt,1);
		\tikzling[xshift=1.5em,yshift=0.1mm,scale=0.4,graduate=black,tassel=red]
		\node[] at (\linewidth/2,0.5) {#1};
		\node at (\linewidth-2em,0.5) {\footnotesize\if@link \hyperlink{mokuji\thechapter}{\color{magenta}章の最初に戻る}\else \fi};
	\end{tikzpicture}
}{-7pt}{}[]
\titleformat{name=\subsection,numberless}[hang]{\normalfont\large\bfseries}{}{0pt}{#1}[]
\titlespacing{\section}{0pt}{4mm}{2mm}
}
%
\DeclareOption{toc=normal}{
\renewcommand{\tableofcontents}[1][c]{%
\hypertarget{mokuji}{}\noindent\makebox[\linewidth][#1]{\LARGE\bfseries 目次}
\bigskip\@starttoc{toc}\bigskip}
}
%
\DeclareOption{toc=1}{
\renewcommand{\PartialToc}{\normalsize\normalfont\begin{shapebox2}[yellow]{目次}"Contents"[drop fuzzy shadow,breakable]
		\startcontents[chapters]
		\printcontents[chapters]{}{1}[2]{}\end{shapebox2}\vspace{-3em}}
\renewcommand{\tableofcontents}[1][orange]{%
\hypertarget{mokuji}{}\begin{shapebox2}[#1]{目次}"Contents"[drop fuzzy shadow,breakable]
\@starttoc{toc}\end{shapebox2}}
}
%
\DeclareOption{partialtoc}{\@ptctrue
\setcounter{tocdepth}{0}}
%
\ExecuteOptions{toc=normal}
\ProcessOptions\relax
%
\titlecontents{chapter}[0em]{\smallskip\large\bfseries}{\makebox[4em][l]{\thecontentslabel}}{}{\if@ptc\quad\titlerule*[1em]{・}\quad\thecontentspage\else\hfill\thecontentspage\fi}[\smallskip]
\titlecontents{section}[1em]{}{\makebox[2.5em][l]{\thecontentslabel}}{}{\quad\titlerule*[1em]{・} \thecontentspage}
\titlecontents{subsection}[3.5em]{}{\makebox[3em][l]{\thecontentslabel}}{}{\quad\titlerule*[1em]{・} \thecontentspage}
\titlecontents{subsubsection}[6.5em]{}{\makebox[4em][l]{\thecontentslabel}}{}{\titlerule*[1em]{・} \thecontentspage}

\newcommand{\crefrangeconjunction}{{〜}}%
\newcommand{\crefrangepreconjunction}{}%
\newcommand{\crefrangepostconjunction}{}%
\newcommand{\crefpairconjunction}{, }%
\newcommand{\crefmiddleconjunction}{, }%
\newcommand{\creflastconjunction}{, }%
\newcommand{\crefpairgroupconjunction}{, }%
\newcommand{\crefmiddlegroupconjunction}{, }%
\newcommand{\creflastgroupconjunction}{, }%

\crefname{equation}{式}{式}%
\crefname{figure}{図}{図}%
\crefname{subfigure}{図}{図}%
\crefname{table}{表}{表}%
\crefname{subtable}{表}{表}%

\crefname{appendix}{付録}{付録}%
\crefname{subappendix}{付録}{付録}%
\crefname{subsubappendix}{付録}{付録}%
\crefname{subsubsubappendix}{付録}{付録}%

\crefname{chapter}{第}{第}
\creflabelformat{chapter}{#2#1章#3}
\crefname{section}{第}{第}
\creflabelformat{section}{#2#1節#3}
\crefname{subsection}{第}{第}
\creflabelformat{subsection}{#2#1節#3}
\crefformat{equation}{(#2#1#3)}
\crefformat{page}{#2#1ページ#3}%
\crefrangeformat{page}{#3#1#4{〜}#5#2#6{ページ}}%
\crefmultiformat{page}{#2#1#3{ページ}}{, #2#1#3{ページ}}{, #2#1#3}{, #2#1#3{ページ}}%
%\crefformat{section}{#2#1#3{節}}%
\crefrangeformat{section}{#3#1#4{〜}#5#2#6{節}}%
\crefmultiformat{section}{#2#1#3{節}}{, #2#1#3{節}}{, #2#1#3}{, #2#1#3{節}}%
\crefformat{subsection}{#2#1#3{節}}%
\crefrangeformat{subsection}{#3#1#4{〜}#5#2#6{節}}%
\crefmultiformat{subsection}{#2#1#3{節}}{, #2#1#3{節}}{, #2#1#3}{, #2#1#3{節}}%
\crefformat{subsubsection}{#2#1#3{節}}%
\crefrangeformat{subsubsection}{#3#1#4{〜}#5#2#6{節}}%
\crefmultiformat{subsubsection}{#2#1#3{節}}{, #2#1#3{節}}{, #2#1#3}{, #2#1#3{節}}%

\newpagestyle{ps=2}[]{
	\sethead{}{}{}
	\setfoot{}{\raisebox{-1em}{\begin{tikzpicture}
			\tikzling[scale=0.5,speech={\thepage}]
	\end{tikzpicture}}}{}
}
\newpagestyle{ps=1}[]{
	\sethead{\begin{tikzpicture}[remember picture,overlay]
			\fill[top color=green!60!] (current page.north east) rectangle ([yshift=-\paperwidth/8]current page.north west);
			\fill[bottom color=yellow!60!,top color=white] (current page.south east) rectangle ([yshift=\paperwidth/8]current page.south west);
	\end{tikzpicture}}{\S\,\thesection\quad\sectiontitle}{}
	\setfoot{}{\large\thepage}{}
}
\newpagestyle{normalps}[]{
	\sethead{}{\thesection~~\sectiontitle}{}
	\headrule\footrule
	\setfoot{}{\thepage}{}
}

\newcommand*{\ctext}[1]{\raise0.2ex\hbox{\textcircled{\scriptsize{#1}}}}
\newcommand*{\sfrac}[2]{{}^{#1}\!/_{#2}}
\newcommand{\dd}[3]{
	\if 1#1 \frac{\mathrm{d}#2}{\mathrm{d}#3}\else{%
		\frac{\mathrm{d}^{#1}#2}{\mathrm{d}#3^{#1}}}\fi
}
\newcommand{\del}[3]{
	\ifcat a#1\frac{\partial^2 #1}{\partial#2\partial#3}\else{%
		\ifnum #1>1\frac{\partial^{#1} #2}{\partial #3^{#1}}\else{%
			\ifnum #1=1 \frac{\partial #2}{\partial #3}\else{%
				\text{エラー} (最初の引数は1以上の整数) }\fi
		}\fi%
	}\fi
}
\newcommand{\paren}[1]{\left(#1\right)}
\newcommand{\curly}[1]{\left\{#1\right\}}
\newcommand{\bracket}[1]{\left[#1\right]}
\newcommand{\abs}[1]{\left|#1\right|}
